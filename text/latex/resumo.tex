Esta disserta��o prop�e-se a analisar o problema de classifica��o de textos no contexto de redes sociais (mais especificamente o Facebook). Inicialmente foram implementadas Multinomial Na�ve Bayes, utilizando como feature as palavras dos textos, e posteriormente foram inclu�dos metadados das postagens. Como os resultados n�o foram satisfat�rios, explorou-se um modelo de Weighted Na�ve Bayes com pesos baseados na Teoria da Informa��o, o que trouxe melhoras significativas na qualidade dos resultados.

A partir do classificador criado, desenvolveu-se uma extens�o para o navegador Chrome que modifica o feed de not�cias do Facebook adicionando um cabe�alho a cada postagem que contem o seu assunto. O usu�rio pode modificar o assunto, caso discorde da classifica��o automatizada, dando um feedback que permite uma aprendizagem online para as Redes Bayesianas constru�das.