%This work analyses the problem of text classification in the context of social networks (Facebook, specifically). Different versions of Multinomial Na�ve Bayes were implemented using only words as features at first, then some post's metadata too. Since the results were not satisfying, a Weighted Na�ve Bayes model was implemented with weights based on Information Theory, which significantly improved the classifier's performance.
%Afterwards, a chrome extension, that modifies the News Feed adding headers to each post with a corresponding tag, was developed. The user can modify this tag if he disagrees with the automatic classification and his feedback allows the Bayesian Network to learn online.

Social networks are increasingly present in people's daily lives helping them communicate and keeping them informed. However, this popularization also brings the problem of unwanted information or posts, which takes the user's time and makes the experience unpleasant. The problem of text classification in social networks is analyzed in this context. A Multinomial Na�ve Bayes classifier was implemented using text words as features. The classifier was integrated to Facebook using an extension for the Chrome browser. This extension modifies the post, adding a header with information about its classification / topic. The user can also change the topic if he disagrees with the one provided by the plugin. In this case, this information feeds the Bayesian Networks that were built allowing them to learn online. In a second version, post metadata were included as well. The classifier was trained and evaluated with a posts' database classified by third parties. The Weighted Na�ve Bayes method was explored with the goal of improving the classifier, whose weights were based on Information Theory. This technique brought significant improvements in results quality, increasing the accuracy in the many tested scenarios.